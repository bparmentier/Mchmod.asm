\documentclass[french,10pt,A4]{report}
%\usepackage[latin1]{inputenc}
\usepackage[utf8]{inputenc}
\usepackage[francais]{babel}
% Modification des marges ------------------------------
\oddsidemargin -4mm % Decreases the left margin by 4mm
\textwidth 17cm %Sets text width across page = 17cm
\textheight 22cm %Sets text height up and down = 22cm
% -----------------------------------------------------
\usepackage[T1]{fontenc}
\usepackage{graphicx,color, caption2}
\usepackage{epsfig}
\usepackage{fancyhdr}
\pagestyle{fancy}
\usepackage{listings}
\usepackage{color}
\usepackage{makeidx}
\usepackage{textcomp} % pour le copyleft

% Valeurs par défaut le lstset
\lstset{language={},%C,Assembleur, TeX, tcl, basic, cobol, fortran, logo, make, pascal, perl, prolog, {}
    literate={â}{{\^a}}1 {ê}{{\^e}}1 {î}{{\^i}}1 {ô}{{\^o}}1 {û}{{\^u}}1
         {ä}{{\"a}}1 {ë}{{\"e}}1 {ï}{{\"i}}1 {ö}{{\"o}}1 {ü}{{\"u}}1
         {à}{{\`a}}1 {é}{{\'e}}1 {è}{{\`e}}1 {ù}{{\`u}}1 
         {Â}{{\^A}}1 {Ê}{{\^E}}1 {Î}{{\^I}}1 {Ô}{{\^O}}1 {Û}{{\^U}}1
         {Ä}{{\"A}}1 {Ë}{{\"E}}1 {Ï}{{\"I}}1 {Ö}{{\"O}}1 {Ü}{{\"U}}1
         {À}{{\`A}}1 {É}{{\'E}}1 {È}{{\`E}}1 {Ù}{{\`U}}1,
    commentstyle=\scriptsize\ttfamily\slshape, % style des commentaires
    basicstyle=\scriptsize\ttfamily, % style par défaut
    keywordstyle=\scriptsize\rmfamily\bfseries,% style des mots-clés
    backgroundcolor=\color[rgb]{.95,.95,.95}, % couleur de fond : gris clair
    framerule=0.5pt,% Taille des bords
    frame=trbl,% Style du cadre
    frameround=tttt, % Bords arrondis 
    tabsize=3, % Taille des tabulations
%   extendedchars=\true, % Incompatible avec utf8 et literate
    inputencoding=utf8,
    showspaces=false, % Ne montre pas les espaces 
    showstringspaces=false, % Ne montre pas les espaces entre ''
    xrightmargin=-1cm, % Retrait gauche 
    xleftmargin=-1cm, % Retrait droit
    escapechar=°}  % Caractère d'échappement, permet des commandes latex dans la source
% -----------------------------------------------------
%\makeindex
\begin{document}
\lhead{Labo F.S. S.E 2$^eme$}
\rhead{Page \thepage}
\lfoot{\textcopyleft Bruno Parmentier (G38496) }
\rfoot{\today}
\cfoot{ }
\renewcommand{\footrulewidth}{0.4pt}

\setlength{\parindent}{0pt} % pas d'indentation

\lstset{frame=trBL}

\setcounter{tocdepth}{1}    % limiter les nivaux de table des matières
\setcounter{secnumdepth}{5} % La numérotation des sections au maximum

\newcommand{\titre}{Titre du sujet} % la variable contenant le titre du sujet

\thispagestyle{empty}

\title{\emph{Laboratoire\\\textbf{F.S.}}}
\author{Bruno Parmentier (G38496)}
\date{13 février 2014}
\maketitle
%\tableofcontents
% ~\\[5cm]
% \flushright{Dis, papa, ça veut dire quoi "Formating drive c" ?}
% \flushleft{ }
\newpage
%
\lstset{language=[x86masm]{Assembler}}
\renewcommand{\titre}{\textcolor{blue}{ FS078 - chmod en assembleur }}

\lhead{ \titre }
\section{{\titre} }

\begin{tabular}{|l|l|}
\hline
Titre : 	& \titre \\\hline
Support : 	& MDV2007 Installation Classique \\\hline
Date :		& 02/2014 \\\hline
\end{tabular}

\subsection{Énoncé}

Réécrivez la commande chmod en asm et baptisez-la Mchmod. Votre Mchmod doit
fonctionner avec un ou plusieurs fichiers donnés en paramètre. Il ne doit gérer
que la forme octale donnée comme premier paramètre.

\subsection{Une solution}

\lstinputlisting{SOURCES/Mchmod.asm}

\subsection{Commentaires}

\begin{itemize}
\item La difficulté du projet résidait principalement dans la conversion du mode
    entré en premier argument. Cette valeur est récupérée sur la pile en ASCII,
    et doit être convertie en octal afin d'exécuter l'appel système chmod.
\item Tous les cas ne sont pas vérifiés. On suppose que le premier argument est
    le mode donné en format octal et les arguments suivants sont les chemins des
    fichiers.
\end{itemize}
%\newpage
\end{document}
